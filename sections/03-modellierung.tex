\chapter{Grundlagen der Modellierung}
\section{Automaten}
\begin{tcolorbox}
Ein \textbf{Standardautomat} $A=(Z, \Sigma, \delta, z_0, Z_F)$ ist beschrieben
durch die Menge der Zustände $Z$, die Menge der möglichen Ereignisse
$\sigma \in \Sigma$, die Zustandsübergangsfunktion $\delta = Z \times \Sigma \to Z$,
den Anfangszustand $z_0$ und der Menge der Endzustände $Z_F$.
\end{tcolorbox}

\section{Petrinetze}
\begin{tcolorbox}
Ein Petrinetz ist ein gerichteter, bipartiter (zwei Knotentypen) Graph
$PN=(P,T,F,K,M)$, der aus Stellen $p\in P$ und Transitionen $t \in T$ gebildet wird.

Stellen und Transitionen sind jeweils über die Kanten $f \in F$ der Flussrelation
miteinander verbunden. Jede Stelle $p_i$ ist mit einer Kapazität $k(p) \in K$ und
einer Anfangsmarkierung $m(p) \in M_0$ annotiert. Jede Transition ist mit einem
Knotengewicht $w(t)\in W$ annotiert.

Knoten sind entweder Transistionen oder Stellen und Kanten verbinden immer
Stellen mit Kanten bzw. Kanten mit Stellen, nie Knoten gleichen Typs.
\end{tcolorbox}

\begin{tcolorbox}
Den \textbf{Vorbereich} $\cdot t$ einer Transition $t$ bilden alle Stellen von
denen eine Kante zu der Transition führen:
\begin{equation}
    \cdot t = \{y | (y,t) \in F\}
\end{equation}

Den \textbf{Nachbereich} $t \cdot$ einer Transitition bilden alle direkten Nachfolger:
\begin{equation}
    t \cdot = \{y | (t, y) \in F\}
\end{equation}
\end{tcolorbox}

\begin{tcolorbox}
    Ein Petrinetz heißt \textbf{schaltbereit}, wenn mindestens eine Transition
    schaltbereit ist. Eine Transition $t \in T$ ist schaltbereit, falls die entsprechenden
    Marker geschaltet werden können ohne Kapazitätsgrenzen zu verletzen.
\end{tcolorbox}

\begin{tcolorbox}
    Ein Petrinetz ohne schaltbereiten Transitionen nennt man ein totes Petrinetz,
    im Gegensatz dazu wird ein Petrinetz mit schaltbereiten Transitionen als
    lebendiges Petrinetz bezeichnet
\end{tcolorbox}

Petrinetze können genutzt werden um ein Deadlock in einem System zu erkennen.
