\chapter{Extrinsische Analyse}
\section{Ablaufplanung nach Fristen}
\subsection{Echtzeittest}
Bei der Ablaufplanung nach Fristen wird die Task geplant, deren spezifiziert Ablauffrist
als erstes erreicht wird.

Jede Task wird durch $c^+$ (maximale Nettoausführungszeit) und $d$ (Deadline) beschrieben,
zudem wird die Anregung durch $\hat{\Theta}$ beschrieben.

Die Angeforderte Rechenzeit ist dann gegeben durch:
\begin{equation}
    \rho(\Delta t, \tau) = \eta(\Delta t, \theta_\tau) c^+_\tau
\end{equation}
verschiebt man die Rechenzeitanforderung um die Deadline erhält man die Zeit,
die der Prozessor zu diesem Zeitpunkt abgearbeitet haben muss:
\begin{equation}
    \delta(\Delta t, \tau) = \rho(\Delta t - d_\tau, \tau)
\end{equation}
folglich muss stets gelten:
\begin{equation}
    \delta(\Delta t, \tau) \leq \Delta t
\end{equation}

dieser Test lässt sich auf eine Menge unabhängiger Tasks erweitern:
\begin{equation}
    \delta(\Delta t, \Gamma) = \sum_{\tau \in \Gamma} \rho(\Delta t, \tau)
\end{equation}
auch hier muss gelten:
\begin{equation}
    \delta(\Delta t, \Gamma) \leq \Delta t
\end{equation}
das ist der Test über die Rechenzeitanforderungsdichte (demand bound function)

\subsection{Rechenzeitanforderungsdichte und Echtzeitnachweis für das periodische
Ereignismodell}
Für das periodische Ereignismodell ist die Ereignisdichte gegeben durch:
\begin{equation}
    \eta^+(\Delta t, \hat{\theta}^+) = \floor*{\frac{\Delta t}{p} + 1}
\end{equation}

\begin{tcolorbox}
    \textbf{Rechenzeitanforderungstest für periodische Tasks} Sei $\Gamma$ eine
    Menge unabhängiger, periodischer, unterbrechbarer Tasks, die auf einem Prozessor 
    mit EDF geplant werden. Diese Taskmenge hält alle gesetzte Fristen ein, falls gilt:
    \begin{equation}
        \delta^+(\Delta t, \Gamma) = \sum_{\tau \in \Gamma} 
            \floor*{\frac{\Delta t - d_\tau}{p_\tau} + 1} c^+_\tau \leq \Delta t\ \ 
        \forall \Delta t: \min_{\tau \in \Gamma}(d_\tau) \leq \Delta t \leq h
    \end{equation}
    dabei ist $h$ die Hyperperiode, also das kleinste gemeinsame Vielfache aller
    Perioden.
\end{tcolorbox}

für den Fall das Frist gleich der Periode ist vereinfacht sich die Bedingung zu:
\begin{equation}
    \sum_{\tau \in \Gamma} \floor*{\frac{\Delta t}{p_\tau}} c_\tau^+ \leq \Delta t
\end{equation}
durch abschätzen der Gaußklammer durch ihr Argument erhält man die Auslastungsanalyse
\begin{tcolorbox}
    \textbf{Auslastungsanalyse:} Sei $\Gamma$ eine Menge unabhängiger, unterbrechbarer,
    periodischer Tasks die auf einem Prozessor nach EDF geplant werden. Unter der Annahme,
    dass alle Fristen gleich der Periode sind hält diese Taskmenge die gesetzten Fristen
    ein falls gilt:
    \begin{equation}
        \sum_{\tau \in \Gamma} \frac{c^+_\tau}{p_\tau} \leq 1
    \end{equation}
    die Summe wird auch als Auslastung $U$ bezeichnet.
\end{tcolorbox}

\section{Effizienter Echtzeittest}
Es wird betrachtet für welche $\Delta t$ die Rechenzeitanforderungsanalyse
ausgewertet werden muss. Da $\delta$ eine stückweise konstante Funktion mit
Sprüngen ist reicht es die Funktion an den Sprungstellen auszuwerten:
\begin{equation}
    \Delta = \bigcup_{\tau \in \Gamma} \left\{t | t = np_\tau + d_\tau, n \in \mathbb{N}^+,
        t < h\right\}
\end{equation}
offensichtlich gilt $\abs{\Delta} \in \mathcal{O}(\exp(\abs{\Gamma}))$.

Für einen effizienten Test müssen also Approximationen eingesetzt werden.

\subsection{Verkürzung der Testgrenze}
\begin{tcolorbox}
    \textbf{Testgrenze nach Baruah:} Es ist ausreichend den Echtzeittest bis zur
    Testgrenze:
    \begin{equation}
        h' = \frac{U}{1-U} \max_{\tau \in \Gamma} (p_\tau - d_\tau)
    \end{equation}
    durchzuführen
\end{tcolorbox}

Beweisansatz: Annahme: Frist wird verletzt, 
Definition Rechenzeitanforderungsanalyse, Gaußklammern abschätzen,
$p_\tau - d_\tau$ durch Maximum abschätzen, $\delta$ durch $\Delta t$ nach unten abschätzen
nach $\Delta t$ auflösen.

\begin{tcolorbox}
    \textbf{Testgrenze nach Ripoll: } Es ist ausreichend, den Echtzeittest bis zur
    Testgrenze:
    \begin{equation}
        h'' = \frac{\sum_{\tau \in \Gamma} (p_\tau - d_\tau) \frac{c_\tau}{p_\tau}}
                {1-U}
    \end{equation}
    durchzuführen.
\end{tcolorbox}
Beweisansatz: Gaußklammer der Rechenzeitanforderungsanalyse nach oben abschätzen,
ergibt Geradengleichung mit positivem y-Achsenabschnitt und Steigung ($=U$) kleiner 1,
ist also ab $\Delta t'$ garantiert kleiner als $\Delta t$, Schnittpunkt bestimmen,
neue Testgrenze.

\section{Approximation}
Problem: bei hoch ausgelasteten Systemen sind alternativen Testgrenzen nicht kürzer.

\subsection{Approximation durch Straffen der Periode}
Idee: einzelne Perioden verkürzen, macht Test strenger aber kann Hyperperiode drastisch
kürzen.

\begin{tcolorbox}
Sei $\delta(\Delta t, \Gamma)$ die Rechenzeitdichte und $\delta'(\Delta t, \Gamma)$ die
approximierte Rechenzeitdichte, dann ist der Approximationsfehler:
\begin{equation}
    \varepsilon = \max_{\delta t \in [0, h]} \left(\frac{\delta'(\Delta t, \Gamma)-
        \delta(\Delta t, \Gamma)}{\delta(\Delta t, \Gamma)} \right)
\end{equation}
\end{tcolorbox}
Problem: das finden der optimalen Straffung ist wiederrum nicht effizient lösbar.

\subsection{Approximation durch konstante Intervalle}
Idee: Nicht mehr jeden Testpunkt untersuchen sondern $n$ äquidistante Punkte. Dafür
müssen alle Sprünge des folgenden Segments nach vorne ziehen.

\begin{tcolorbox}
Das Testintervall $h$ wird in $n$ Intervalle der Länge $\Delta=\ceil*{\frac{h}{n}}$
unterteilt. Dann berechnet sich die Rechenzeitdichte zu:
\begin{equation}
    \delta'(\Delta t, \Gamma) = \sum_{\tau \in \Gamma} \floor*{
        \frac{\Delta t + \Delta - \delta_\tau}{p_\tau} + 1} c_\tau^+
\end{equation}
\end{tcolorbox}

Der Fehler ist angeblich gegeben durch:
\begin{equation}
    \varepsilon \leq \frac{\ceil*{\frac{h}{n}}}{n \ceil*{\frac{h}{n}} +
        \sum_{\tau \in \Gamma} (p_\tau - d_\tau)}
\end{equation}

\subsection{Approximation durch Superposition}
Idee: die Rechenzeitdichte der einzelnen Tasks wird nur für die ersten Sprünge
genau berechnet, danach durch eine gerade angenähert, die Gesamtrechenzeitdichte
ergibt sich durch Superposition der einzelnen Taskrechenzeitdichten.

\begin{tcolorbox}
    \textbf{Approximierte Rechenzeitdichte einer Task:} Die Rechenzeitdichte einer
    Task kann für $k$ Testpunkte exakt berechnet werden und wird dann durch die
    mittlere Auslastung der Tasks approximiert:
    \begin{equation}
        \delta'(k, \Delta t, \tau) = \begin{cases}
            \delta(\Delta t, \tau) & \Delta t \leq k p_\tau + d_\tau \\
            \delta(kp_\tau+d_\tau, \tau) + \frac{c^+_\tau}{p_\tau}
            (\Delta t - (k p_\tau + d_\tau)) & \text{sonst}
        \end{cases}
    \end{equation}
\end{tcolorbox}
Beweis: zeigen das $\delta'(k, \Delta t, \tau) \geq \delta(\Delta t, \tau)$.

\begin{tcolorbox}
    Der \textbf{Fehler der Approximation durch Superposition} ist umgekehrt
    Proportional zur Anzahl der Testpunkte:
    \begin{equation}
        \varepsilon \leq \frac{1}{k}
    \end{equation}
\end{tcolorbox}

\subsection{Ablaufplanung nach Prioritäten}
Ansatz: Antwortzeitanalyse.

\begin{tcolorbox}
    \textbf{Antwortzeitanaylse:} Die Antwortzeit einer Task $\tau_i$ in einem
    prioritätengesteuerten System ist die maximale Nettoausführungszeit $c^+$
    der Task addiert mit dern maximalen Nettoausführungszeiten aller Tasks mit einer
    höheren Priorität als die betrachtete Task, $\tau \in \Gamma_\text{HP}$.

    Die Antwortzeit ist dann gegeben durch:
    \begin{equation}
        r_i^+(n) = c_i^+ + \sum_{\tau \in \Gamma_\text{HP}} \check{\eta}(
            r_i^+(n-1), \hat{\Theta}_j) c_\theta^+
    \end{equation}
    mit
    \begin{equation}
        r^+_i(0) = c_i^+
    \end{equation}
    dabei wird die Rekursion bis zur Konvergenz berechnet.
\end{tcolorbox}

\section{Modulare Analyse verteilter Systeme}

