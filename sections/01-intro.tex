\chapter{Einführung}
\section{Historische Entwicklung}
\begin{itemize}
    \item 1928, 1936 mechanisches ABS
    \item Zweiter Weltkrieg: V2/A4
    \item 1938: Z1, erster Programmierbarer Rechner
    \item 1947: Transistor
    \item ICBMs (Minuteman)
    \item 1962: Mariner 2, Venus
    \item 1969: Mondlandung
    \item 1964: ICs
    item 1978: Elektronisches ABS
\end{itemize}

\section{Definition eines eingebetteten Systems}
Ein eingebettetes System ist ein Rechner oder Computer, der in ein andere technisches System zur Steuerung, Regelung
oder Überwachung eingebettet ist.

Anforderungen:
\begin{itemize}
    \item Zuverlässigkeit
    \item Raue Umgebung
    \item Vorhersagbarkeit
    \item Teilweise: Energieverbrauch/Leistungsaufnahme
\end{itemize}

Eingebettete Systeme, die ihre Aufgaben rechtzeitig, also innerhalb vorgegebener Fristen erledigen müssen, nennt man
Echtzeit- ooder Realzeitsysteme.

\begin{tcolorbox}
Ein \textbf{System} ist eine Menge untereinander in Wechselbeziehung stehender Elemente,
Objekte, Komponenten oder Module. Dabei erfüllt ein System einen vorgegebenen Zweck.
Das System erfüllt seine Aufgabe meist durch Umwandlung und Verarbeitung von eingehende
in ausgehenede Stoff-, Energie- oder Informationsflüsse. Zwischen den einzelnen
Elementen des Systems bestehen ebenfalls Wechselbeziehungen aus Stoff-, Energie- und
Informationsflüssen.
\end{tcolorbox}

\begin{tcolorbox}
Ein \textbf{Prozess} ist ein Satz von Wechselbeziehungen oder Tätigkeitne, die Eingaben
(\glqq{}Signale\grqq{}) in Ergebnisse wandeln
\end{tcolorbox}

\begin{tcolorbox}
Ein \textbf{eingebettetes System} ist ein Rechner oder Computer, der in einem 
übergeordneten technischen System einen oder mehrere physikalischen Prozesse kontrolliert.
Der Rechner ist somit in ein technisches System eingebunden oder eingebettet.
Dabei reagiert er reaktiv auf Signale und erzeugt nach erfolgter Beechnung wieder Signale,
um das technische System zu kontrollieren.
\end{tcolorbox}

\section{Definition der Zeit}
Motivation: Veränderung (DGLs), Rechtzeitig, Ordnung/Koordination von Ereignissen.

\begin{tcolorbox}
\textbf{Zeit} ist ein Konstrukt, um Bewegungen n Theorie und Experimenten beschreiben zu
können.
\end{tcolorbox}

\begin{tcolorbox}
Eine \textbf{Uhr} ist ein Messgerät, um Zeit zu messen. Sie besteht aus einem periodischen
physikalischen Prozess und einem Zählwerk, das die Anzahl der Schwingungen zählt. Der
Zählstand wird zu jeder Zeit angezeigt.
\end{tcolorbox}

Historisch:
\begin{itemize}
    \item Zeit über Tag (aber nicht gleich lang): Sonne als Periodischer Oszillator aber zu grob.
    \item 1590: Galileo: Fallexperimente mit Puls als Zeit
    \item 1640: Galileo: Pendeluhr
    \item 1687: Newton: Infinitesimalrechnung (Bewegung als Zeitliche Änderung/DGL)
    \item Heute: UTC über radioaktive Zerfallsprozesse; Elektronische Oszillatoren (Quarz)
\end{itemize}

\subsection{Messung langer Zeitabstände}
Problem: Es wird ein \glqq{}ewiger Zähler\grqq{} benötigt.

Alternativen:
\begin{itemize}
    \item Natürlicher Zähler: Baumringe (nach oben beschränkt)
    \item Radioaktive Zerfallsprozesse (Halbwertszeit)
\end{itemize}

\subsection{Navigation durch Zeit}
Direkt der Breitengrad über Winkel zwischen Horizont und Sonne (plus Datumskompensation).

Längengrad kann nicht einfach gemessen werden (\glqq{}Längengradproblem\grqq{}).

Ansätze:
\begin{itemize}
    \item Koppelnavigation/Odometrie (ungenau)
    \item Distanz zu Mond/Gestirne (genaue Messinstrumente benötigt, auf schwankendem Schiff schwierig)
    \item Galileo: Jupitermonde als Uhr
    \item Kanonenschiffe für Zeitsynchronisierung
    \item 1760: Chronometer (genau, mechanische Uhr)
    \item Heute: GPS (weiter basierend auf Zeit, mit vier Satelliten aber keine genaue Uhr notwendig)
\end{itemize}

\section{Zeit in eingebetteten Systemen}
\begin{tcolorbox}
    \textbf{Echtzeitbetrieb} ist der Betrieb eines Rechnersystems, bei dem Programme zur 
    Verarbeitung anfallender Daten ständig betriebsbereit sind, derart, dass die
    Verarbeitungsergebnisse innerhalb einer vorgegebenen Zeitspanne verfügbar sind.
    Die Daten können ja nach Anwendungsfall nach einer zufälligen, zeitlichen Verteilung
    oder zu bestimmten Zeitpunkten auftreten.
\end{tcolorbox}

\subsection{Steuerflussdominante Echtzeitsysteme}
Werden eingebettete Rechner eingesetzt, um physikalische Vorgänge eines technischen
Prozesses zu steuern, erhält man ein steuerflussdominanten Echtzeitsystem.

Bsp.: Motorsteuerung.

\subsection{Datenflussdominante Echtzeitsysteme}
Setzt man eingebettete Rechner ein, um die Signalauswertung zu realisieren, erhält man ein
Datenflussdominantes Echtzeitsystem.

Bsp.: Sonar (Aus dem Skript: SAR, Erklärung zu komplex).

\subsection{Steuerfluss- und datenflussrelevante Echtzeitsysteme}
Prozesssteuerung und Signalverarbeitung gleichermaßen wichtig.

Bsp: Mobilfunk.
