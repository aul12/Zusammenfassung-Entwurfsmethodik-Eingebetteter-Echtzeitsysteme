\chapter{Extrinsische Modelle}
\section{Datenflussmodelle}
Ein markierter Graph ist ein Datenflussgraph bei dem die einzelnen Knoten
vollständige Tasks sind.

\begin{tcolorbox}
    Ein \textbf{markierter Graph} ist ein gerichteter Graph. Die Menge $V$ der Knoten
    entspricht Tasks oder Aufgaben des zu modellierenden Systems. Die Kanten
    $E \subseteq V \times V$ des Graphens modellieren den Datenfluss von einer Aufgabe zur
    nächsten, der Vorgang des Datenflusses wird durch eine Marke beschrieben.

    Die Kanten des Graphen entsprechen dabei Speichern nach FIFO-Strategie.
    Ein markierter Graph kann atomare Zustandswechsel durchführen, dabei
    konsumiert und produziert ein Knoten in jedem Schritt der Zustandsfolge jeweils
    eine Marke.
\end{tcolorbox}

\begin{tcolorbox}
    Ersetzt man die Knoten eines markierten Graphen durch Transitionen und die Kanten
    durch Stellen, dann entspricht der \textbf{markierte Graph} einem \textbf{Petrinetz}
    mit den folgenden Eigenschaften:
    \begin{enumerate}
        \item Der Vor- und Nachbereich jeder Stelle enthält jeweils nur eine Transistion
        \item Beliebig viele Marken können auf den Kanten des Graphen gespeichert werden
        \item Jeder Knoten des Graphen kann genau eine Marke produzieren bzw. konsumieren
    \end{enumerate}
\end{tcolorbox}

Besondere Eigenschaften:
\begin{itemize}
    \item Die Summe der Marken in einem gerichteten Zyklus ist konstant
    \item Wenn die Summe der Marken in jedem Zyklus des Graphens positiv ist,
        so ist der markierte Graph stets lebendig
\end{itemize}

Bei Systemen, bei denen sich im Verlauf der Berechnung die Raten der Taskaktivierung
ändern nennt man Multiratensysteme.

\begin{tcolorbox}
    Ein \textbf{synchroner Datenflussgraph} ist ein gerichteter Graph.
    Dabei ist $V$ die Menge aller Knoten (Aktivitäten oder Aufgaben),
    $E \subseteq V \times V$ die Menge aller Kanten (FIFO Speicher).
    Das Attribut $\text{cons}: E \to \mathbb{N}^+$ entspricht die Anzahl der
    aus dem FIFO-Speicher einer Kante konsumierten Marken, $\text{prod}: E 
    \to \mathbb{N}^+$ spezifiziert die Anzahl der anfänglich in den Speichern
    vorhandenen Marken.
\end{tcolorbox}

\begin{tcolorbox}
    Eine \textbf{Topologiematrix} eines synchronen Datenflussgraphen wird wie folgt 
    gebildet:
    Verläuft eine Kante $e_j$ von Knoten $v_i$ nach Knoten $v_k$, dann sind alle
    Einträge der Spalte $c_j$ bis auf $c_{ij}=-\text{prod}(v_i, v_k)$ und
    $c_{kj}=-\text{prod}(v_i, v_k)$ gleich Null.
\end{tcolorbox}

\begin{tcolorbox}
    Ersetzt man die Knoten des \textbf{synchronen Datenflussgraphen} durch Transistionen,
    die Kanten durch Stellen, so erhält man ein äquivalentes \textbf{Petrinetz}. Mit folgenden
    Eigenschaften:
    \begin{itemize}
        \item Der Vor- und Nachbereich einer Stelle enthält jeweils nur ein Transistion.
        \item Die Kapazität der Stellen ist unbegrenzt
        \item Jeder Eingangskante einer Transition wird ein Kantengewicht
            $\text{cons}(t,p) = w(p,t)$ und jeder Ausgangskante einer Transition
            ein Kantengewicht $\text{prod}(t,p) = w(t,p)$ zugewiesen.
    \end{itemize}
\end{tcolorbox}

\subsection{Taskgraphen}
\begin{tcolorbox}
    Eine \textbf{Task} $\tau \in \Gamma$ ist definiert durch folgende Struktur
    \begin{equation}
        \tau = (c^+, c^-, d)
    \end{equation}
    mit der maximalen Nettoausführungszeit $c^+$, der minimalen Nettoausführungszeit
    $c^-$ und der Zeitfrist $d$.
\end{tcolorbox}

\begin{tcolorbox}
    Ein Taskgraph $G_\tau(\Gamma, E)$ ist ein Graph, der Knoten $\tau \in \Gamma$ und
    Kanten $e \in E \subseteq \Gamma \times \Gamma$. Jeder Kante $e \in E$ ist dabei ein
    Kantengewicht $\Theta$ zugeordnet.

    Das Kantengewicht nennt man auch Ereignissequenz ind beschreibt den zeitlichen
    Mindestabstand zwischen Ereignissen auf dem Kommunikationskanal zwischen $\tau_i$
    und $\tau_j$.
\end{tcolorbox}

\section{Ereignissequenzen}
\begin{tcolorbox}
    Eine \textbf{Ereignissequenz} ist eine zeitliche Abfolge von Ereignissen und wird
    mit $\Theta$ bezeichnet.
\end{tcolorbox}

Die Anzahl von Ereignissen pro Zeiteinheit soll Ereignisdichte heißen. Wird dabei das
maximal mögliche Auftreten von Ereignissen so spricht man von einer maximalen 
Ereignisdichte $\eta^+(t)$, beschreibt man das minimal mögliche Auftreten von Ereignissen,
spricht man von der minimalen Ereignisdichte $\eta^-(t)$.

\begin{tcolorbox}
    Sei $[t_1, t_2]$ ein Zeitintervall, der Länge $t=t_2-t_1$ und $\Theta$ eine
    Ereignissequenz, dann bestimmt die \textbf{begrenzte Ereignisdichte} $\eta: \mathbb{R}
    \to \mathbb{N}$ die Summe aller in disem Zeitintervall aufgetretenen Ereignisse.

    Dabei sind $t_1$ und $t_2$ eingeschlossen.
\end{tcolorbox}

\begin{tcolorbox}
    Es gilt:
    \begin{equation}
        \eta^+(t) \geq \eta(t) \geq \eta^-(t)
    \end{equation}
\end{tcolorbox}

\begin{tcolorbox}
    Die \textbf{unbegrenzte Ereignisdichte} ist die begrenzte Ereignisdichte
    bis zu $t_2 - \varepsilon$ für $\varepsilon \to 0$ und wird mit 
    $\check{\eta}$ bezeichnet.
\end{tcolorbox}

Im folgenden wird Ereignisdichte die begrenzte Ereignisdichte bezeichnet.

\subsection{Periodische Ereignissequenz}
Ein Ereignis wird periodisch, mit zeitlichem Abstand $p$ unendlich oft wiederholt.

Formal:
\begin{equation}
    \Theta = \{\Theta_p\} \text{ mit } \Theta_p = p
\end{equation}

Die begrenzte Ereignisdichte ist gegeben durch:
\begin{equation}
    \eta(t) = \floor*{\frac{t}{p}} + 1
\end{equation}

Die unbegrenzte Ereignisdichte durch:
\begin{equation}
    \check{\eta} = \ceil*{\frac{t}{p}}
\end{equation}

\subsection{Periodische Ereignissequenz mit Jitter}
Erweiterung des periodischen Modells bei dem Ereignisse um bis zu$\pm j$ verschoben sein
können.

Formal:
\begin{equation}
    \Theta = \{\Theta_\text{jitter}\} = \{(p, j)\}
\end{equation}

Im Mittel ist die Ereignisdichte gleich der Ereignisdichte der Periodischen 
Ereignissequenz.

Die maximale Ereignisdichte ist jedoch gegeben durch:
\begin{equation}
    \eta^+(t) = \floor*{\frac{t+2 j}{p} + 1}
\end{equation}

\subsection{Sporadisch, periodische Ereignissequenz}
Periodisch ($p_0$) treten $n$ Ereignisse als Burst mit Periode $p_i$ auf.

Formal:
\begin{equation}
    \Theta = \{\Theta_\text{sporadic}\} = \{(p_i, p_0, n)\}
\end{equation}

Die maximale Ereignisdichte lässt sich (sehr konservativ) durch die
Dichte der inneren, periodischen, Ereignissequenz abschätzen.

\subsection{Periodische Ereignissequenz mit initialem Schub}
Die Ereignissequenz besteht aus einem initialen Schub gefolgt von einer periodischen
Ereignissequenz.

Es gilt $j>p$. Die äußere Periode ist $p$, die innere $a$.
\begin{equation}
    \Theta = \{\Theta_\text{burst}\} = \{(p, j, a)\}
\end{equation}

Die Ereignisdichte ist gegeben durch:
\begin{equation}
    \eta(t) = \min\left(\floor*{\frac{t+j}{p}} + 1, \floor*{\frac{t}{a}} + 1\right)
\end{equation}

\section{Ereignisströme}
Erweiterung für beliebige Strukturen.

\begin{tcolorbox}
Ein \textbf{Ereignisstrom} ist eine Ereignissequenz $\Theta$, in der alle Ereignisse
mit ihrem kürzesten Abstand auftreten. Beschrieben wird ein Ereignisstrom durch eine
Folge von Ereignistupeln $\theta$. Jedes Tupel besteht aus einer Periode $p$, die
es ermöglicht Wiederholungen von Ereignissequenzen zu beschreiben und einem Zeitintervall
$a$, welches den minimalen Abstand, der auf dieses Tupel bezogenen Ereigniszahl festlegt.

Der Abstand $a_i$ im $i$-ten Tupel bestimmt demnach die Dichte von $i$ Ereignissen:
\begin{equation}
    \Theta = \{\theta_i\} = \left\{ \binom{p}{a}_i\right\}
\end{equation}
\end{tcolorbox}

Alternative Beschreibung:
\begin{itemize}
    \item Bestimme $\eta^+(\Delta t)$ durch Sliding Window
    \item Sprünge der resultierenden Treppenform beschreiben:
        $a$ ist Position des ersten Sprungs, $p$ ist Periode (abstand zwischen Sprüngen), Sprung ist jeweils eins hoch
    \item Durch Superposition einzelnen Treppenfunktionen $\eta^+(\Delta t)$
        bestimmen
\end{itemize}

Die maximale Ereignisdichte berechnet sich durch:
\begin{equation}
    \eta^+(\Delta t, \Sigma) = \sum_{\theta \in \Theta, \Delta t \geq a_\theta}
        \floor*{\frac{\Delta t - a_\theta}{p_\theta}} + 1
\end{equation}

Hinweise:
\begin{itemize}
    \item Die Ereignisdichte ist nach definition die maximale
    \item Es findet implizit eine Abbildung in den Intervallbereich statt
    \item Die jeweils maximale Ereignisdichte kann über ein sliding Window
        Ansatz bestimmt werden
\end{itemize}

\subsection{Maximale Dichte}
\begin{tcolorbox}
    \textbf{Maximale Ereignisdichte: } Sei $\Theta^+$ die zur Anregung $\Theta$
    zugehörige Anregung maximaler Dichte. Die Dichte der maximalen Ereignissequenz
    beschreibt die maximale Anzahl von Ereignissen für ein gegebenes Zeitintervall
    $\Delta t$ aus der Ereignissequenz $\Theta$:
    \begin{equation}
        \eta(\Theta^+, \Delta t) = \max_{t \in \mathbb{R}^+} \left(
        \eta(\Theta, t + \Delta t) - \check{\eta}(\Theta, t)
        \right)
    \end{equation}
\end{tcolorbox}

\begin{tcolorbox}
    \textbf{Monotonie der maximalen Dichte: } Die begrenzte Ereignisdichte der Anregung
    maximaler Dichte $\eta(\Theta^+, \Delta t)$ ist monoton wachsend bezüglich
    $\Delta t$.
\end{tcolorbox}

\subsection{Minimale Dichte}
Die minimale Dichte beschreibt eine Ereignissequenz mit minimaler Ereignisse
pro Zeiteinheit aus der Ereignissequenz $\Theta$. Die Aussagen zur maximalen Dichte
lassen sich analog auch auf die minimale Dichte übertragen.

\subsection{Periodische Ereignissequenz als Ereignisstrom}
Die periodische Ereignissequenz $\Theta = \{p\}$ kann mit einer Folge wiederkehrender,
periodischer Ereignisse mit Abstand $0$ in der Notation des Ereignisstroms beschrieben
werden.

Die periodische Ereignissequenz ist homogen, d.h. die minimale ist gleich der maximalen
Ereignisdichte. D.h. Ereignisstrom und Ereignisdichte sind gleich.

\subsection{Periodische Ereignisequenz mit Jitter als Ereignisstrom}
Der Ereignisstrom ist gegeben durch:
\begin{equation}
    \Theta^+ = \left\{ \binom{\infty}{0}_1, \binom{p}{p-2j}_2 \right\}
\end{equation}

\subsection{Sporadisch periodische Ereignissequenz als Ereignisstrom}
Der Ereignisstrom ist gegeben durch:
\begin{equation}
    \Theta^+ = \left\{ \binom{p_i}{(1-1)p_0}_1, \binom{p_i}{(2-1)p_0}_2, \cdots, \binom{p_i}{(n-1)p_0}_{n-1}, 
        \binom{p_i}{p_0}_n \right\}
\end{equation}

\subsection{Periodische Ereignissequenz mit initialem Schub als Ereignisstrom}
Zuerst muss der Wechselpunkt zwischen initialem Schub und Periodischen Verhalten
bestimmt werden:
\begin{equation}
    n = \ceil*{\frac{j}{p-a}}
\end{equation}

Damit kann der Ereignisstrom analog zum Modell Sporadisch periodisch definiert werden:
\begin{equation}
    \Theta^+ = \left\{
        \binom{\infty}{0}_1,
        \binom{\infty}{a}_2,
        \binom{\infty}{2a}_3,
        \cdots,
        \binom{\infty}{(n-1)a}_n,
        \binom{p}{np-j}_{n+1}
        \right\}
\end{equation}


\section{Ereignisspektrum}
Problem: Ereignisströme werden zu komplex.

Lösung: Hierarchische Ereignisströme.

\begin{tcolorbox}
    Ein \textbf{Ereignisspektrum} $\hat{\Theta}$ beschreibt allgemein die Beziehung
    zwischen dem Auftreten von Ereignissen bezogen auf ein gegebenes Zeitintervall
    $\Delta t$ und erlaubt es, beliebig geschachtelte Ereignissequenzen zu beschreiben.
    
    Sei $\hat{\Theta} = \{\hat{\theta}\}$ eine Menge von Spektralelementen
    \begin{equation}
        \hat{\theta}_i = \left(p_{\hat{\theta}}, a_{\hat{\theta}}, 
            \lambda_{\hat{\theta}}, f_{\hat{\theta}}, \hat{\Theta}_{\hat{\theta}}
            \right) 
    \end{equation}
    mit
    \begin{itemize}
        \item $p$ Abstand der Ereignisse des Tochterspektrums
        \item $a$ Phase
        \item $\lambda$ Anzahl der vom Tochterspektrum erzeugten Ereignisse
        \item $f$ Eingebettete Ereignisrate
        \item $\hat{\Theta}$ Tochterspektrum
    \end{itemize}
\end{tcolorbox}

\begin{tcolorbox}
    Sei $\hat{\theta}_i = \left(p_{\hat{\theta}}, a_{\hat{\theta}}, \lambda_{\hat{\theta}}, f_{\hat{\theta}}, \hat{\Theta}_{\hat{\theta}}\right)$
    ein Ereignisspektrum. Dann spricht man von einem gültigen Ereignisspektrum,
    wenn einerseits einzelne Spektralelemente nicht überlappen und andererseits
    entweder $f=0$ oder $\hat{\Theta} = \emptyset$.
\end{tcolorbox}

\subsection{Umwandlung zu gültigen Ereignisspektren}
\begin{tcolorbox}
    \textbf{Ereignisintervall: } Sei $n$ die Anzahl von Ereignissen und $\theta$ ein
    Ereignisspektralelement, dann berechnet sich das kleinste Interval $\Delta t$ in
    dem $\theta$ $n$ Ereignisse erzeugt werden mit:
    \begin{equation}
        \Delta t(n, \hat{\theta}) = \argmin_{\Delta t \in \mathbb{R}^+}\left(
            \eta(\Delta t, \hat{\theta}) \geq n\right)
    \end{equation}
\end{tcolorbox}

\begin{tcolorbox}
    Jedes Ereignisspektrum kann in ein \textbf{gültiges Ereignisspektrum umgewandelt}
    werden. Das der Separationsbedingung genügende Spektrum errechnet sich wie folgt:
    \begin{eqnarray}
        \hat{\Theta} &=& \{\hat{\theta}_1, \ldots, \hat{\theta}_k\} \\
        \text{mit } \hat{\theta}_i &=& \left(
            k p_{\hat{\theta}}, (i-1) p_{\hat{\theta}} + a_{\hat{\theta}},
            \lambda_{\hat{\theta}}, f_{\hat{\theta}}, \hat{\Theta}_{\hat{\theta}}\right) \\
        \text{und } k &=& \ceil*{\frac{\Delta t(\lambda_{\hat{\theta}}, \hat{\Theta}_{\hat{\theta}})}{p_{\hat{\theta}}}}
    \end{eqnarray}
\end{tcolorbox}

\subsection{Ereignisspektraldichte}
Definition der Ereignisdichte auf Basis von Ereignisspektren.

\begin{tcolorbox}
    \textbf{Ereignisspektraldichte:} Sei für jedes $\Delta t$
    \begin{equation}
        \eta(\Delta t, \hat{\Theta}) = \sum_{\hat{\theta} \in \hat{\Theta}, 
            a_{\hat{\theta}} \leq \Delta t}
            \eta(\Delta t, \hat{\theta})
    \end{equation}
\end{tcolorbox}

\begin{tcolorbox}
    Eine Funktion $f: A \to B$ ist \textbf{subadditiv}
    \begin{equation}
        \Leftrightarrow \forall x, y \in A:
            f(x+y) \leq f(x) + f(y)
    \end{equation} 

    die Funktion heißt \textbf{superadditiv}
    \begin{equation}
        \Leftrightarrow \forall x, y \in A:
            f(x) + f(y) \leq f(x+y)
    \end{equation}
\end{tcolorbox}

\begin{tcolorbox}
    $\hat{\Theta}$ ist ein oberes oder \textbf{maximales Spektrum} $\hat{\Theta}^+$, 
    wenn alle
    Ereignisspektralelemente gültige Elemente sind und wenn die Ereignisdichte
    subadditiv bezüglich $\Delta t$ ist, also gilt:
    \begin{equation}
        \forall \Delta t_1, \Delta t_2 \in \mathbb{R}^+_0:
            \eta(\hat{\Theta}^+ , \Delta t_1 + \Delta t_2) \leq
            \eta(\hat{\Theta}^+ , \Delta t_1) +\eta(\hat{\Theta}^+ , \Delta t_2)
    \end{equation}
    äquivalent dazu sind \textbf{minimale Ereignisspektren} superadditiv.
\end{tcolorbox}

\begin{tcolorbox}
    Jedes Ereignisspektrum kann in ein \textbf{äquivalentes Spektrum minimaler
    Tiefer} umgeformt werden, dabei hat das Ereignisspektrum nur noch eine
    Rekursionstiefe von eins.
\end{tcolorbox}

Ein Ereignisspektrum
\begin{equation}
    \hat{\Theta}_\alpha = \{p_\alpha, a_\alpha, \lambda_\alpha, 0, \hat{\Theta}_\alpha'\}
\end{equation}
mit dem Tochterspektrum
\begin{equation}
    \hat{\Theta}_\alpha' = \{(p_1', a_1', \lambda_1', 0, \hat{\Theta}_1'),
        \cdots (p_k', a_k', \lambda_k', 0, \hat{\Theta}_k')\}
\end{equation}
wird zu
\begin{equation}
    \hat{\Theta}_\beta = \{\hat{\theta}_{\beta, 1}, \cdots, \hat{\theta}_{\beta, k+1}\}
\end{equation}
mit
\begin{equation}
    \hat{\theta}_{\beta, i} = (p_\alpha, a_\alpha, \eta(\Delta t_\alpha, 
        \hat{\theta}_{\alpha,i}'), 0, \hat{\theta}_{\alpha, i}')
\end{equation}
und ($\varepsilon > 0$)
\begin{equation}
    \Delta t_\alpha = \lim_{\varepsilon \to 0} \left(
        \Delta t(\lambda_\alpha, \Theta_\alpha') - \varepsilon\right)
\end{equation}
sowie
\begin{equation}
    \hat{\theta}_{\beta, k+1} = \left(\infty, \Delta t(\lambda_\alpha, \hat{\Theta}_a'),
        \lambda_\alpha - \sum_{\hat{\theta}_\alpha \in \hat{\Theta}_\alpha}
        \eta(\Delta t_\alpha, \hat{\theta}_{\alpha, i}), \infty, \emptyset\right)
\end{equation}

\subsection{Periodische Ereignissequenz als Ereignisspektrum}
\begin{eqnarray}
    \hat{\Theta}^+ &=& \{(p, 0, 1, \infty, \emptyset)\} \\
    \hat{\Theta}^- &=& \{(p, p, 1, \infty, \emptyset)\}
\end{eqnarray}

\subsection{Ereignissequenz mit Jitter als Ereignisspektrum}
Für $j<p$:
\begin{eqnarray}
    \hat{\Theta}^+ &=& \{(p,0,1,\infty, \emptyset), (p, p-2j, 1, \infty, \emptyset)\} \\
    \hat{\Theta}^- &=& \{(p,0,1,\infty, \emptyset), (p, p+2j, 1, \infty, \emptyset)\}
\end{eqnarray}
für $j>p$:
\begin{eqnarray}
    \hat{\Theta}^+ &=& \left\{\left(\infty, 0, \floor*{\frac{j}{p}}, \infty, \emptyset\right), \left(p, p-\left(\frac{j}{p} - \floor*{\frac{j}{p}}\right), 1, \infty, \emptyset\right)\right\} \\
    \hat{\Theta}^- &=& \{(\infty,0,1,\infty, \emptyset), (p, p+2j, 1, \infty, \emptyset)\} 
\end{eqnarray}

\subsection{Periodisch sporadische Ereignissequenz als Ereignisspektrum}
\begin{eqnarray}
    \hat{\Theta}^+ &=& \left\{ \left( p_0, 0, n, 0, \left\{(p_i, 0, 1, \infty, \emptyset)\right\}\right)\right\} \\ 
    \hat{\Theta}^- &=& \left\{ \left( \infty, 0, 1, \infty, \emptyset\right),
        \left(p_0, p_0, n, 0, \left\{ (p_i, 0, 1, \infty, \emptyset) \right\} \right)
        \right\}
\end{eqnarray}

\subsection{Periodische Ereignissequenz mit initialem Schub}
\begin{eqnarray}
    \hat{\Theta} &=& \left\{\left(\infty, 0, \floor*{\frac{j}{p-a}}, 0,
        \left\{(a, 0, 1, \infty, \emptyset)\right\}\right), \left(p, np-j, 1, \infty, \emptyset\right)\right\} \\
    \hat{\Theta}^- &=& \left\{(\infty, 0, 1, \infty, \emptyset), (p, p+j, 1, \infty, \emptyset)\right\} \\
    \hat{\Theta}^+ &=& \left\{\left(\infty, 0, n, 0, \left\{(a,0,1,\infty,\emptyset)\right\}\right), (p, d, 1, \infty, \emptyset)\right\}
\end{eqnarray}
mit
\begin{eqnarray}
    n &=& \floor*{\frac{j}{p}+1} \\
    d &=& n p -j
\end{eqnarray}
