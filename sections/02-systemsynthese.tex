\chapter{Modellgestützte Systemsynthese}
Ziel: Systematischer Entwurf eines möglichst Kostengünstigen Systems das alle 
Randbedingungen einhält.

Ablauf:
\begin{enumerate}
    \item Spezifikation:
        \begin{itemize}
            \item Tools: Spezifikations-, Programmier- oder Hardwarebeschreibungssprache
            \item Anforderungen z.B. Zeitfristen, Energieverbrauch, Kosten
        \end{itemize}
    \item Modell:
        \begin{itemize}
            \item Wird mit Compiler aus Spezifikation gewonnen
            \item Wählt Bauelemente (Hardware und Software) aus Plattformbibliothek
        \end{itemize}
    \item Intrinsische Analyse: 
        \begin{itemize} 
            \item Extrahiert Parameter die abhängig von der Spezifikation und der Bauelemente ist
            \item z.B. Ausführungszeit
        \end{itemize}
    \item Extrinsische Analyse: 
        \begin{itemize}
            \item Das resultierende System wird analysiert
            \item z.B. Echtzeitverhalten, Energieverbrauch, Kosten
        \end{itemize}
    \item Optimierungsverfahren
        \begin{itemize}
            \item Ermittelt Entwurfsqualität aus Systemmodell und vergleicht mit anderen Entwürfen
            \item Kann auch Systemmodell verändern.
        \end{itemize}
\end{enumerate}

\begin{tcolorbox}
    Das \textbf{Entwurfsziel} ist eine Menge verschalteter Hardwarekomponenten, die eine
    gegebene Menge von Aufgaben unter spezifizierten Randbedingungen ausführt. Dabei sind
    die Kosten (Fläche, Energieverbrauch, etc.) des System zu minimieren.
\end{tcolorbox}

\begin{tcolorbox}
    Die \textbf{Spezifikation} enthält alle notwendigen Informationen, um die
    funktionalen Aufgaben und die Randbedingungen zu beschreiben. Dabei kann die
    Spezifikation aus den folgenden Einzelteilen bestehen:
    \begin{itemize}
        \item Operationen und Relationen (Elementare Rechnenvorschriften oder Logikfunktionen)
        \item Befehl (Operation oder Kontrollanweisung)
        \item Programm (Geordnete Menge von Befehlen)
        \item Aufgabe/Task (Transformation einer Menge von Eingabedaten in eine Menge
            von Ausgabedaten)
        \item Prozess (Program + Daten)
    \end{itemize}
\end{tcolorbox}

Richtige Wahl der Modellgenauigkeit wichtig: 
zu genaue Modelle brauche zu lange zum analysieren, zu grobe Modelle ignorieren Details.

Mögliche Modellgenauigkeiten:
\begin{enumerate}
    \item Befehle/Operationen
    \item Basisblöcke
    \item Aufgaben
    \item Prozesse
\end{enumerate}

Die Plattformbilbiothek stellt Kommunikations und Rechenbausteine zur Verfügung. Hinzu
kommen Verwaltungskompnenten z.B. Treiber und Ablaufplanungsverfahren.

\begin{tcolorbox}
    Eine \textbf{Verarbeitungseinheit} ist eine befehlersverarbeitende (ISP oder ASIP)
    oder anwendungsspezifische Hardwarekomponente (ASP) oder eine Kommunikationsstruktur,
    die in einem festgelegten Zeitintervall exakt eine Aufgabe bearbeiten kann.

    Alternativ: Ressource.
\end{tcolorbox}

\begin{tcolorbox}
    Eine \textbf{Ausführungsdomäne} verwaltet die Ressourcen eine Verarbeitungseinheit
    gemäß des spezifizierten Ablaufplanungsverfahrens. Eine Ausführungsdomäne
    beinhaltet mehrere Ausführungsdomänen oder Tasks: die Prozesse.
    Zudem enthält eine Ausführungsdomäne ein Ablaufplanungsverfahren, dass festlegt,
    in welcher Reihenfolge die einzelnen Prozesse abgearbeitet werden.
\end{tcolorbox}

\begin{tcolorbox}
    Unter \textbf{Nettoausführungszeit (Ausführungszeit)} versteht man die Ausführungszeit
    einer Task oder eines Programs, wenn diese die Ressource exklusiv zur Verfügung hat
    und nicht von anderen Tasks unterbrochen wird. Es wird zwischen $c^+$ und $c^-$
    (worst- bzw. best-case execution time) unterschieden.
\end{tcolorbox}

Verfahren zur Bestimmung der Nettoausführungszeit:
\begin{itemize}
    \item Direkte Messung (nicht alle möglichen Pfade, Messung kann Zeit beeinflussen)
    \item Simulation (Instruction Set Simulator, weiterhin nicht alle Pfade)
    \item Probabilistisch (Nur Verteilungsfunktion wird bestimmt)
\end{itemize}

\begin{tcolorbox}
    Die \textbf{Bruttoausführungszeit (Antwortzeit)} entspricht der Zeit, die nach
    Aktivierung der Task gewartet werden muss bis ein Ergebnis vorliegt (Benennung
    $r^+$ und $r^-$).
\end{tcolorbox}

Die Optimierung besteht aus den folgenden Schritten:
\begin{enumerate}
    \item Allokation: Auswahl und Bestimmung der Verarbeitungseinheiten und
        Ausführungsdomänen. 
    \item Bindung: Bestimmung welche Task oder Ausführungsdomäne in welcher
        Ausführungsdomäne oder Verarbeitungseinheit ausgeführt wird.
    \item Ablaufplanung: Die Ablaufplanung spezifiert nach einem vorgegebenen Schema die
        Reihenfolge der Abarbeitung von Aufgaben.
\end{enumerate}

\section{Modellbildung}
Ein Taskgraph besteht aus Tasks als Knoten, gruppiert in Verarbeitungseinheiten.
Der Datenaustausch wird durch (gerichtete) Kanten gezeigt.
In einem Taskgraph können auch Busse als Verarbeitungseinheiten eingefügt werden
und die selben Methoden der Ablaufplanung angewandt werden.

Das Taskmodell beschreibt die Ereignissequenz mit der eine Task angeregt wird.

Zudem kann die Rechenzeit der einzelnen Tasks bestimmt werden.

Nun können die Kanten im Graph mit den entsprechenden Ereignissequenzen beschriftet
werden, die Knoten werden mit der jeweiligen Rechenzeit beschriftet.

\begin{tcolorbox}
    Das \textbf{Systemmodell} zur Echtzeitanalyse und Entwurfsraumexploration
    besteht aus folgenden Teilen:
    \begin{enumerate}
        \item Ereignismodell: Das Ereignismodell beschreibt die zeitliche Anregung der
            Tasks vom technischen System und die Kommunikationshäufigkeit im System.
        \item Taskmodell: Das Taskmodell stellt die Struktur der Anwendung dar. Dabei
            gibt es an welche Aufgaben Datenabhängigkeiten mit anderen Aufgaben haben
            und welche maximale und minimale Rechenzeit eine Aufgabe benötigt.
        \item Prozessmodell: Das Prozessmodell modelliert die Prozess- und
            Hardwarestruktur und gibt an welche Ressourcen wannn genutzt werden können.
    \end{enumerate}
\end{tcolorbox}
