\chapter{Intrinsische Analyse}
Unterscheidung:
\begin{itemize}
    \item Profiling: Aufrufhäufigkeiten statistisch erfasst
    \item Flussanalyse: Längster Programmpfad wird gesucht
\end{itemize}

\section{Prozessorspezifische Schätzung}
Maschinencode wird mit Zielcompiler erstellt, Information aus Profiling wird
genutzt um Zykluszeit zu bestimmen.

\section{Generischer Schätzer}
Compiler Frontend wird genutzt um Prozessorunabhängigen drei-Adresscode zu generiern.
Technologiedatei für den Zielprozessor sammelt Ressourcenverbrauch (Zeit, Speicher)
pro drei-Adresscode. Mithilfe der Informationen aus dem Profiling kann dann die
minimale und maximale Ausführungszeit bestimmt werden.

\section{Schätzung durch Programmpfadanalyse}
Kein ausführen/Simulieren des Programms, die maximale und minimale Nettoausführungszeit
werden durch Fluss- oder Pfadanalyse bestimmt.

Es wird zuerst ein Basisblock oder Kontrollflussgraph erstellt. Die Ausführungszeiten
der Basisblöcke wird als bekannt und konstant angenommen. Jeder Basisblock
wird mit Ausführungszeit $c_i$ und Aufrufhäufigkeit $x_i$ annotiert.

Die minimale und maximale Ausführungszeit kann dann über die Suche nach dem 
kürzesten/längsten Programmpfad bestimmt werden:
\begin{equation}
    c^+ = \max\left(\sum_{i=1}^n c_i x_i \right)
\end{equation}
dafür muss die Anzahl der Schleifendurchläufe ich vorraus bekannt oder zumindest
beschränkt sein.

\section{Strukturelle Randbedingungen}
Beschreiben die Struktur des Basisblockgraphens. Jede Kante wird mit $d$, der
Aufrufhäufigkeit beschrieben.

Für Verzweigungen gilt:
\begin{equation}
    d_\text{Condition} = d_\text{true} + d_\text{false}
\end{equation}

Für Schleifen gilt:
\begin{equation}
    d_\text{Begin} + d_\text{Loop} = d_\text{true} + d_\text{false}
\end{equation}

\section{Funktionale Randbedingungen}
Funktionale Randbedingungen modellieren die Aufrufhäufigkeiten, die sich aus Belegung
von Variablen etc. ergibt.

\section{Mikroarchitekturmodellierung}
Um komplexere Ausführungsmodelle (Cache, Pipelining) zu berücksichtigen
muss die Kostenfunktion erweitert werden, so kann z.B. $c_i$ in $c_i^\text{miss}$
und $c_i^\text{hit}$ mit dazugehörigem $x_i^\text{miss}$ und $x_i^\text{hit}$.
