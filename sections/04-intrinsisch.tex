\chapter{Intrinsische Modelle}
\section{Datenflussgraph}
Ein Datenflussgraph besteht aus Knoten $v \in V$, die Rechenoperationen, Aufgaben
oder Ähnliches modellieren, und Kanten $E \subseteq V \times V$, die den Datenfluss
zwischen den Operationen beschreiben.

Ein Datenflussgraph kann auch als Petrinetz beschrieben werden. Hierbei sind
Stellen (Kanten des Datenflussgraphs) Speicher für Daten.

\section{Grundblock- und Kontrollflussgraph}
\begin{tcolorbox}
    Ein \textbf{Grundblock} eines Programms ist ein Folge von Befehlen,
    die den Programmfluss nicht verändern, d.h. ein Grundblock ist ein
    Befehlssegment, in dem der Programmfluss nicht verzweigt. Begrenzt wird ein
    Grundblock durch Sprung- oder Verzweigungsbefehle.
\end{tcolorbox}

\begin{tcolorbox}
    Ein \textbf{Grundblockgraph} ist ein gerichteter, zyklischer Graph $G(V, E)$ mit Knoten
    die den Grundblöcken eines Programms entsprechen und Kanten, die den Kontrollfluss
    eines Programms modellieren. Die Kanten können noch mit Verzweigungsbedingungen
    annotiert werden.
\end{tcolorbox}

\begin{tcolorbox}
    Ein \textbf{Kontrollflussgraph} ist ein gerichteter, bipartiter, zyklischer
    Grundblockgraph. Dabei unterscheidet man Steuerblöcke und Basisblöcke.
\end{tcolorbox}

\section{Kontroll-/Datenflussgraph}
Die einzelnen Grundblöcke eines Kontrollflussgraphens können durch Datenflussgraphen
modelliert werden. Ein Kontrollflussgraph mit hierarchisch untergeordneten
Datenflussgraphen nennt man Kontroll-/Datenflussgraph.
